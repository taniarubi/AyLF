\documentclass[letterpaper,12pt]{article}

% Soporte para los acentos.
\usepackage[utf8]{inputenc}
\usepackage[T1]{fontenc}    
% Idioma español.
\usepackage[spanish,mexico, es-tabla]{babel}
% Soporte de símbolos adicionales (matemáticas)
\usepackage{multirow}
\usepackage{amsmath}		
\usepackage{amssymb}		
\usepackage{amsthm}
\usepackage{amsfonts}
\usepackage{latexsym}
% Modificamos los márgenes del documento.
%\usepackage[lmargin=2cm,rmargin=2cm,top=2cm,bottom=2cm]{geometry}
% Soporte para dibujar con Tikz.
\usepackage{tikz}
\usetikzlibrary{automata, positioning, arrows}

% Información para el título
\title{Autómatas y Lenguajes Formales\\ Resumen sobre Classic Nintendo games are (computationally) hard}
\author{Rodríguez Torres Víctor Fidel\\ Rubí Rojas Tania Michelle}
\date{29 de noviembre de 2018}
\begin{document}
\maketitle
Consideramos que el tema de complejidad computacional nos interesó aún más ya que no sabíamos que se podía hacer un análisis tan profundo a estos problemas en el tema de los videojuegos, no nos imaginabamos que en los videojuegos era un problema, computacional, saber si un jugador podía llegar de un punto inicial a un punto objetivo.\\

Nos pareció peculiar que se utilizará el problema 3-SAT para hacer las reducciones, sería interesante ver alguna demostración en la cual se utilizará una reducción distinta a 3-SAT para los problemas tratados en el artículo. Nos gustó la reducción del problema de los cuantificadores de la FOL para el problema de las puertas, la manera en que se atacó el problema de las puertas con el problema de los cuantificadores fue muy ingenioso.Fue impresionante la manera en que se hicieron las demostraciones, es muy interesante la manera profesional en que los autores hacen las demostraciones, cuando eramos niños nunca nos imaginabamos que un juego que se ve simple, llegase a ser computacionalmente complejo, ahora vemos los videojuegos de manera distinta, valoramos más el trabajo de los programadores de videojuegos y el análisis computacional que se debe hacer para ver si un jugador puede llegar a un punto, de tal manera en que esto tiene un gran impacto en la creación de los escenarios en cada nivel.\\

Nos gustó que el profesor abarcara el tema de complejidad, de esta manera podmeos entender mucho mejor este tipo de artículos. 




\end{document}


