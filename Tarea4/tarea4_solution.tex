\documentclass[letterpaper,10pt]{article}

% Soporte para los acentos.
\usepackage[utf8]{inputenc}
\usepackage[T1]{fontenc}    
% Idioma español.
\usepackage[spanish,mexico, es-tabla]{babel}
% Soporte de símbolos adicionales (matemáticas)
\usepackage{multirow}
\usepackage{amsmath}		
\usepackage{amssymb}		
\usepackage{amsthm}
\usepackage{amsfonts}
\usepackage{latexsym}
\usepackage{enumerate}
% Modificamos los márgenes del documento.
\usepackage[lmargin=2cm,rmargin=2cm,top=2cm,bottom=2cm]{geometry}
% Soporte para dibujar con Tikz.
\usepackage{tikz}
\usetikzlibrary{automata, positioning, arrows}

% Información para el título
\title{Autómatas y Lenguajes Formales\\ Tarea 4}
\author{Rodríguez Torres Víctor Fidel\\ Rubí Rojas Tania Michelle}
\date{29 de noviembre de 2018}

\begin{document}
 \maketitle
 
 \begin{enumerate}
     
     % Ejercicio 1.
     \item Demuestra que
     \begin{itemize}
         
         % Demostración 1.1
         \item Si $M$ es una Máquina de Turing determinista con $k$ cintas que
         corre en tiempo $O(f(n))$, existe una Máquina de Turing determinista
         $M'$ con una única cinta y equivalente a $M$ que corre en tiempo 
         $O(f(n)^{2})$.
         
         \begin{proof}
             Supongamos que $M$ es una Máquina de Turing determinista con 
             $k$ cintas que corre en tiempo $O(f(n))$. Construyamos una Máquina
             de Turing determinista $M'$  con una única cinta que sea
             equivalente a $M$ y corra en tiempo $O(f(n)^{2})$. \\
             Sabemos que $M'$ es capaz de simular a $M$ gracias al siguiente
             teorema visto en clase: 
             
             $\newtheorem{teo}{Teorema}$
             
             \begin{teo}
                 Toda Máquina de Turing multicinta tiene una Máquina de Turing
                 con una única cinta equivalente.
             \end{teo}

             Para revisar esta simulación, recordemos que $M'$ usa una  
             única cinta para representar los contenidos que hay en todas las
             $k$ cintas de $M$. Estas cintas se almacenan consecutivamente,
             con las posiciones de las cabezas de $M$ marcadas en las casillas
             apropiadas. \\ 
             Inicialmente, $M'$ pone su cinta en el formato que representa
             todas las cintas de $M$ y luego simula los pasos de $M$. Para
             simular un paso, $M'$ escanea toda la información almacenada
             en su cinta para determinar los símbolos bajo las cabezas de cinta
             de $M$. Luego, $M'$ hace otro escaneo sobre su cinta para  
             actualizar el contenido de la cinta y las posiciones de la cabeza.
             Si una de las cabezas de $M$ se mueve hacia la derecha sobre la
             parte previamente no leída de su cinta, $M'$ debe aumentar la 
             cantidad de espacio asignado a esta cinta. Lo hace desplazando
             una parte de su propia cinta una celda a la derecha.\\
             Ahora, analizamos esta simulación. Para cada paso de $M$, la
             máquina $M'$ realiza dos escaneos sobre la parte activa de su
             cinta. El primero obtiene la información necesaria para determinar
             el siguiente movimiento y el segundo lo lleva a cabo. La longitud
             de la parte activa de la de la cinta de $M'$ determina cuánto
             tiempo se tarda en escanearla, por lo que debemos determinar
             un límite superior en esta longitud. Para ello, tomamos la suma de
             las longitudes de las partes activas de las $k$ cintas de $M$.
             Cada una de estas porciones activas tiene la longitud a lo más de
             $f(n)$ porque $M$ utiliza $f(n)$ celdas de cinta en $f(n)$ pasos
             si la cabeza se mueve hacia la derecha en cada paso, e incluso
             menos si una cabeza se mueve siempre a la izquierda. Por lo tanto,
             una exploración de la parte activa de la cinta $M'$ utiliza
             $O(f(n))$ pasos. \\
             Para simular cada uno de los pasos de $M$, $M'$ realiza dos
             escaneos y posiblemente hasta $k$ desplazamientos hacia
             la derecha. Cada uno utiliza el tiempo $O(f(n))$, por lo que el 
             tiempo total para que $M'$ simule uno de los pasos de $M$ es 
             $O(f(n))$. Ahora, limitamos el tiempo total utilizado por la
             simulación. La etapa inicial, donde $M'$ pone su cinta en el
             formato adecuado, utiliza $O(n)$ pasos. Después, $M'$ simula
             cada uno de los $f(n)$ pasos de $M$, utilizando $O(f(n))$ pasos,
             por lo que esta parte de la simulación utiliza $f(n) x O(f(n)) $
             $= O(f(n)^{2})$ pasos. Por lo tanto, toda la simulación de $M$
             usa $O(n) + O(f(n)^{2})$ pasos. \\
             Hemos asumido que $f(n) \geq n$ (ya que $M$ ni siquiera podía leer 
             la entrada completa en menos tiempo). Por lo tanto, el tiempo de
             ejecución de $M'$ es $O(f(n)^{2})$. 
             
         \end{proof}
         
         \newpage
         
         % Demostración 1.2
         \item Si $M$ es una Máquina de Turing no-determinista que corre en 
         tiempo $O(f(n))$, existe una Máquina de Turing determinista $M'$
         equivalente a $M$ que corre en tiempo $O(2^{f(n)})$.
         
         \begin{proof}
             Supongamos que $M$ es una Máquina de Turing no-determinista
             que corre en tiempo $O(f(n))$. Construyamos una Máquina de
             Turing determinista $M'$ que simule a $M$ como en la prueba
             del siguiente teorema visto en clase:
             
             \newtheorem{Teo}{Teorema}
             \begin{teo}
                 Toda Máquina de Turing no determinista tiene una Máquina
                 de Turing determinista equivalente.
             \end{teo}
             
             buscando el árbol de cálculo no determinista de $M$. En una 
             entrada de longitud $n$, cada rama del árbol de cómputo no
             determinista de $M$ el árbol tiene una longitud de a lo más
             $f(n)$. Cada nodo del árbol puede tener a lo más $k$ hijos,
             donde $k$ es el número máximo de opciones legales dadas
             por la función de transición de $M$. Por lo que, el número
             total de hojas en el árbol es a lo más $k^{f(n)}$. \\
             La simulación procede explorando primero la amplitud de este 
             árbol. En otras palabras, visita todos los nodos en profundidad
             $p$ antes de pasar a cualquiera de los nodos en profundidad
             $p + 1$. El algoritmo del teorema mencionado anteriormente
             comienza en la raíz y viaja hacia un nodo cada vez que visita
             ese nodo.  El número total de nodos en el árbol es menor que
             el doble del número máximo de hojas, por lo que lo
             limitamos por $O(k^{f(n)}$. El tiempo que se tarda en arrancar
             desde la raíz y viajar hasta un nodo es $O(f(n))$. Por lo tanto,
             el tiempo de ejecución de $M'$ es $O(f(n)k^{f(n)}) = O(2^{f(n)})$.
             
         \end{proof} 

     \end{itemize}
     
     % Ejercicio 2.
     \item Demuestra que
     
     \begin{itemize}
         
         % Demostración 2.1
         \item La clase $P$ es cerrada bajo unión, concatenación. ¿Es cerrada
         bajo la operación de complemento? Justifica tu respuesta.
         
         \begin{proof}
             Sean $L_1$ y $L_2$ lenguajes. Veamos que la clase $P$ es cerrada
             bajo unión. Supongamos que $L_1, L_2 \in P$. Entonces existen
             $M_1$ y $M_2$ Máquinas de Turing que corren en tiempo 
             polinomial y deciden a $L_1$ y $L_2$, respectivamente. En
             particular, digamos que $M_1$ tiene el tiempo de ejecución
             $O(n^{i})$ y $M_2$ tiene el tiempo de ejecución $O(n^{j})$, donde
             $n$ es la longitud de la entrada $w$ y $i, j$ son constantes. \\
             La Máquina de Turing $M_3$ que decide $L_1 \cup L_2$ es la
             siguiente: \\ \\
             En la entrada $w$:
             
             \begin{itemize}
                 
                 \item[i)] Ejecuta $M_1$ con la entrada $w$. If $M_1$ acepta, 
                 entonces acepta.
                 \item[ii)] Ejecuta $M_2$ con la entrada $w$. If $M_2$ acepta,
                 entonces acepta. \\
                 En otro caso, rechaza.
                 
             \end{itemize}
             
             Así, $M_3$ acepta la cadena $w$ si y sólo si $M_1$ ó $M_2$
             (o ambos) aceptan a $w$. El tiempo de ejecución de $M_3$ es
             $O(n^{i}) + O(n^{j}) = O(n^{max(i, j)})$, que sigue siendo
             polinomial en $n$, i.e., la suma de dos polinomios también es
             polinomial. Por lo tanto, la clase $P$ es cerrada bajo la 
             operación unión. \\ \\
             Ahora, vemos que la clase $P$ es cerrada bajo la concatenación.
             Supongamos que $L_1, L_2 \in P$. Entonces existen
             $N_1$ y $N_2$ Máquinas de Turing que corren en tiempo 
             polinomial y deciden a $L_1$ y $L_2$, respectivamente. En
             particular, digamos que $N_1$ tiene el tiempo de ejecución
             $O(n^{k})$ y $N_2$ tiene el tiempo de ejecución $O(n^{l})$, donde
             $n$ es la longitud de la entrada $w$ y $k, l$ son constantes. \\
             La Máquina de Turing $N_3$ que decide $L_1 \cup L_2$ es la
             siguiente: \\ \\
             En la entrada $w = a_1a_2...a_n$, con cada $a_i \in \Sigma$ un 
             símbolo:
             
             \begin{itemize}
                 \item[i)] Para $i = 0, 1, 2, ..., n$ hacer
                 
                 \begin{itemize}
                     \item Ejecuta $N_1$ con la entrada $w_1 = a_1a_2...a_i$
                     y ejecuta $N_2$ con la entrada $w_2 = a_{i+1}a_{i+2}...a_n$
                     . Si ambas máquinas $N_1$ y $N_2$ aceptan, entonces
                     acepta.
                 \end{itemize}
                 
                 \item[ii)] Si ninguna de las iteraciones en el paso anterior
                 acepta, entonces rechaza.

             \end{itemize}
             
             La Máquina de Turing $N_3$ verifica todas las formas posibles de 
             dividir la entrada $w$ en dos partes: $w_1$, $w_2$; y verifica si
             la primera parte $w_1$ es aceptada por $M_1$ (i.e. $w_1 \in L_1$)
             y si la segunda parte $w_2$ es aceptada $M_2$ (i.e. $w_2 \in L_2$)
             , por lo que la concatenación de $w_1w_2 \in L_1L_2$. Supongamos
             que la entrada $w$ en $N_3$ tiene una longitud $|w| = n$. El
             primer paso de $N_3$ se ejecuta a lo más $n + 1$ veces. Cada vez
             que se ejecuta el paso uno, $N_1$ y $N_2$ se ejecutan en cadenas
             $w_1$ y $w_2$ con $|w_1| \leq |w| = n$ y $|w_2| \leq |w| = n$.
             Así, ejecutando $N_1$ en $w_1$ toma un tiempo de ejecución de
             $O(n^{k})$ y ejecutar $M_2$ en $w_2$ toma un tiempo de ejecución
             de $O(n^{l})$, por lo que el paso uno corre en tiempo 
             $O(n^{k}) + O(n^{l}) = O(n^{max(k, l)})$, que es polinomial en $n$.
             Como el paso 1 se ejecuta a lo más $n + 1$ veces, obtenemos que 
             el tiempo de ejecución de $N_3$ es $O(n + 1)O(n^{max(k, l)}) = $
             $O(n^{1 + max(k, l)})$, que es polinomial en $n$. Por lo que, 
             el tiempo de ejecución de $N_3$ es polinomial en $n$. Por lo tanto,
             la clase $P$ es cerrada bajo la concatenación. \\ \\
             Finalmente, veamos que la clase $P$ es cerrada bajo el complemento.
             Supongamos que $L_1 \in P$. Entonces existe una Máquina de 
             Turing $S$ que corre en tiempo polinomial que decide $L_1$. La
             Máquina de Turing $S_2$ que decide $S^{c}$ es la siguiente: \\ \\
            En la entrada $w$:
            
            \begin{itemize}
                \item Ejecuta $S_1$ con la entrada $w$. \\ Si $S_1$ acepta,
                entonces rechaza. En otro caso, acepta.
            \end{itemize}
            
            La Máquina de Turing $S_2$ sólo genera lo contrario de lo que hace
            $S_1$, por lo que $S_2$ decide $L_1^{c}$. Por lo que, debido a que
            $S_1$ corre en tiempo polinomial, entonces el tiempo de ejecución
            de $S_2$ también es polinomial. Por lo tanto, la clase $P$ es 
            cerrada bajo la operación complemento. 
            
         \end{proof}

         
         % Demostración 2.2
         \item La clase $NP$ es cerrada bajo unión y concatenación. ¿Es cerrada
         bajo la operación de estrella? Justifica tu respuesta.
         
         \begin{proof}
             Sean $L_1$ y $L_2$ lenguajes. Veamos que la clase $NP$ es cerrada 
             bajo la operación unión. Supongamos que $L_1, L_2 \in NP$. Para 
             cada $i = 1, 2$, sea $V(w, c)$ un algoritmo que, para una cadena
             $w$ y un posible certificado $c$, verifica si $c$ es realmente un 
             certificado para $w \in L_i$.  Así, $V_i(w, c) = 1$ si el 
             certificado $c$ comprueba $w \in L_i$, y $V_i(w, c) = 0$ en otro
             caso.  Por hipótesis  sabemos que $V_i(w, c)$ termina en tiempo
             polinomial $O(|w|^{k})$, para alguna constante $k$. \\
             Para mostrar que $L_1 \cup L_2 \in NP$ construiremos un 
             verificador $V_3$ de tiempo polinomial para la unión de los dos
             lenguajes. Dado que un certificado $c$ para la unión tendrá la 
             propiedad de que $V(w, c) = 1 $ ó $V_2(w, c) = 1$, podemos 
             construir fácilmente un verificador $V_3(w, c) = V_1(w, c) $ 
             $\vee$ $V_2(w, c)$. Claramente, entonces $w \in L_1 \cup L_2$
             si y sólo si hay un certificado $c$ tal que $V_3(w, c) = 1$.
             Notemos que el nuevo verificador $V_3$ se ejecutará en tiempo
             $O(2(|w|^{k}))$, que es polinomial. Por lo que, 
              $L_1 \cup L_2 \in NP$. Por lo tanto, la clase $NP$ es cerrada bajo
              la unión.  \\ \\
              Ahora, veamos que la clase $NP$ es cerrada bajo la operación
              concatenación.  Supongamos que los lenguajes  $L_1, L_2 \in NP$
              con los verificadores $V_1$ y $V_2$ descritos en la prueba de la
              parte anterior. Nuevamente, nuestro objetivo es construir un 
              verificador de tiempo polinomial $V_4(w, c)$ para una cadena $x$
              y el posible certificado $c$. Supongamos que $|w| = n$. Podemos
              definir que $V_4(w, c) = 1$ si y sólo si $c = a\# y \# z$, donde
              $\#$ es un nuevo símbolo, $a \in \{0, 1, ..., n\}$, y
              
              \begin{center}
                  $V_1(w_1 ... w_a, y) = 1$ y $V_2(w_{a+1} ... w_n, z) = 1$
              \end{center}
              
              Notemos que $a$ especifica la posición donde la cadena original 
              $w$ debería dividirse en dos partes, y $y, z$ son los certificados
              para las dos partes. El verificador $V_4$ se ejecutará en tiempo
              $O(|w|^{k})$ desde $|w_1 ... w_a| \leq |w|$ y $|w_{a+1} ... w_n|$
              $\leq |w|$. También, $V_4(w, x) = 1$ si y sólo si $w$ pertenece
              a la concatenación entre $L_1$ y $L_2$. Por lo que la
              concatenación entre $L_1, L_2 \in NP$ está en $NP$. Por lo tanto,
              la clase $NP$ es cerrada bajo la concatenación. \\ \\
              Finalmente, veamos que la clase $NP$ es cerrada bajo la operación
              estrella.  Supongamos que el lenguaje  $L \in NP$. Entonces 
              existe una Máquina de Turing no-determinista $M$ tal que decide
              a $L$ de manera no determinista en tiempo polinomial. La siguiente
              Máquina de Turing no-determinista $M'$ decide a $L^{*}$ de
              manera no determinista en tiempo polinomial: \\ \\
              $M'(w) = $
              
              \begin{itemize}
                  
                  \item[i)] Si $w = \epsilon$, entonces acepta. 
                  \item[ii)] En otro caso, de manera no determinista,
                  particionar la cadena $w$ de cualquier forma entre $1$ a 
                  $|w|$ partes. 
                  \item[iii)] Usar $M$ para comprobar que cada una de las partes
                  está en $L$. Si alguna parte no lo es, entonces rechaza. 
                  
              \end{itemize}
              
              Es sencillo comprobar que esto se ejecuta en tiempo polinomial. 
              Un hecho que es clave es que en cada hilo de computación usamos
              $M$ unicamente hasta $|w|$ veces (una cantidad lineal de veces).
              Como $M'$ decide a $L^{*}$ en tiempo polinomial de manera no
              determinista, entonces podemos concluir que la clase $NP$ es 
              cerrada bajo la operación estrella. 
              
         \end{proof}

     \end{itemize}
     
     % Ejercicio 3.
     \item Muestra que los siguientes problemas están en $P$:
     
     \begin{itemize}
         
         % Demostración 3.1
         \item $ALL_{AFD} = \{\langle M \rangle$ $|$ $M$ es un $AFD$ tal que
         $L(M) = \Sigma^{*}\}$
         
         \begin{proof}
             Sea $M$ un $AFD$. Tenemos que averiguar si $M$ acepta todas las
             cadenas de $\Sigma^{*}$. Notemos que es fácil comprobar si $M$
             rechaza alguna cadena. Para ello realizamos una búsqueda de 
             amplitud o profundidad (BFS o DFS) en el $AFD$ desde el estado inicial, lo
             cual nos toma tiempo polinomial. Si se puede alcanzar un estado de
             no aceptación en $M$, seguramente hay una cadena que no
             pertenece a $L(M)$, por lo que $L(M) \notin \Sigma^{*}$ y 
             $\langle M \rangle \notin ALL_{AFD}$, y si sólo se puede acceder a 
             los estados de aceptación en $M$, entonces $L(M) \in \Sigma^{*}$
             y $\langle M \rangle \in ALL_{AFD}$. \\ 
             Como acabamos de describir un algoritmo de tiempo polinomial
             de manera determinista para  decidir $ALL_{AFD}$, entonces 
             podemos concluir que $ALL_{AFD} \in P$.
             
         \end{proof}

         % Demostración 3.2
         \item $\{\langle G \rangle$ $|$ $G$ es una gráfica simple tal que $G$
         es conexa\}
         
         \begin{proof}
             Sea $L = \{\langle G \rangle$ $|$ $G$ es una gráfica simple tal
             que $G$ es conexa\}. Construyamos una Máquina de Turing $M$
             que decide a $L$ en tiempo polinomial: \\ \\
             $M = $
             En la entrada $\langle G \rangle$:
             
             \begin{itemize}
                 \item[i)] Elegir el primer nodo de $G$ y marcarlo.
                 \item[ii)] Repetir el siguiente paso hasta que no se marquen
                 nuevos nodos:
                 \item[iii)] Para cada nodo en $G$, marcarlo si es vecino de un
                 nodo que ya está marcado.
                 \item[iv)] Escanear todos los nodos de $G$ para determinar si
                 están marcados. Si lo están, acepta. En otro caso, rechazar.
                 
             \end{itemize}
             
             Ahora, veamos que este algoritmo se ejecuta en tiempo polinomial.
             El paso $1$ se ejecuta en tiempo constante, es decir, $O(1)$. El
             paso $2$  ejecuta a lo más $(n + 1)$ repeticiones (ya que, a 
             excepción de la última repetición, cada repetición marca al menos
             un nodo adicional). El paso $3$ se ejecuta en $O(n^{2})$ (ya que
             hay que examinar todos los vecinos del nodo actual para ver si
             alguno ha sido marcado). El paso $4$ se ejecuta en tiempo lineal,
             es decir, $O(n)$ (ya que escanea todos los nodos). Por lo que, el
             algoritmo de ejecuta en $O(n^{3})$. \\
             Como hemos mostrado un algoritmo que se ejecuta en tiempo 
             polinomial de manera determinista que decide a $L$, entonces
             podemos concluirque $L \in P$. \\

         \end{proof}

     \end{itemize}
     
     % Ejercicio 4.
     \item Demuestra que los siguientes problemas están en $NP$. Debes 
     usar la definición de certificados y la definición que ocupa Máquinas
     de Turing no-deterministas:

     \begin{itemize}
         
         % Demostración 4.1
         \item $\{\langle n \rangle$ $|$ $n \in \mathbb{N}$ y existen 
         $a, b \in \mathbb{N}$ números primos tales que $n = ab\}$
         
         \begin{itemize}
             
         	\item Usando definición de certificado.
            
         		\begin{proof}
         			Sea $V$ una MT total con entrada $\langle n,c \rangle$ donde $n\in \mathbb{N}$ y potencialmente puede estar en $A$,el lenguaje, además $c$ está descrito como $c=\langle a,b \rangle$, de tal forma que $a,b \in \mathbb{N}$.\\
         			
         			$V$ hace lo siguiente:
                    
         			\begin{enumerate}[1)]
                        
         				\item Si es verdad que se cumple $esPrimo(a)$ y si se cumple $esPrimo(b)$ se pasa a 2), en otro caso se devuelve false.
         				
         				\item Se devuelve true si se cumple que $n=ab$, se devuelve false en otro caso.
         			
                    \end{enumerate}
         			
                    Como se hizo la certificación en tiempo $O(n)$, llegamos a
                    que $A\in NP$. 
         		
                \end{proof}
         	
         	\item Usando definición de MT no determinista
            
     			\begin{proof}
     				Sea $N$ una MT no determinista con entrada $\langle n \rangle$.\\
     				
     				$N$ hace los siguiente:
                    
     				\begin{enumerate}[1)]
                        
     					\item Asigna a $a$ un número natural seleccionado al azar en el rango $[1,\infty)$.
     					
     					\item Asigna a $b$ un número natural seleccionado al azar en el rango $[1,\infty)$.
     					
     					\item Sea $c=\langle a,b \rangle$, ejecuta $V\langle a,b \rangle$, si $V$ acepta entonces $N$ devuelve true, en otro caso devuelve false.
     				
                    \end{enumerate}
     				
                    Como dimos una MT no determinista que decide a $A$ en tiempo
                    $O(n)$, llegamos a que $A\in NP$.
                    
     			\end{proof}
     			    
     	\end{itemize}
         	
         
         % Demostración 4.2
         \item $\{\langle S, S' \rangle$ $|$ $S$ es un conjunto finito de 
         números enteros, $S' \subseteq S$  y $\sum_{x \in S'} x = 0\}$
         
         \begin{itemize}
             
         	\item Usando definición de certificado.
         	
            \begin{proof}
         		Sea $V$ una MT total con entrada $\langle n,c \rangle$, donde $n=\langle s,s' \rangle$, con $s,s'\subseteq\mathbb{Z}$ y $c = 0$. 
         		
         		$V$ hace lo siguiente:
         		\begin{enumerate}[1)]
                    
         			\item Verifica que se cumpla que $s'\subseteq \mathbb{Z}$, si no se cumple se devuelve false.
         			
         			\item Recorre cada elemento en $s'$ y lo suma con los demás en $s'$, si la suma final es igual a $c$ devuelve true, en otro caso devuelve false.
         		
                \end{enumerate}
         		
                Como se hizo la certificación en tiempo $O(n)$, llegamos a que
                $A\in NP$. 
         	
            \end{proof}
         	
         	\item Usando definición de MT no determinista
         	
            \begin{proof}
         		Sea $N$ una MT no determinista con entrada $\langle s,s' \rangle$.\\
         		
         		$N$ hace los siguiente:
                
         		\begin{enumerate}[1)]
                    
         			\item Define $a=|s|$ y a $b=|s'|$.
         			
         			\item Toma $a$ números al azar en $\mathbb{Z}$ y los guarda en el conjunto $S_1$
         			
         			\item  Toma $b$ números al azar en $\mathbb{Z}$ y los guarda en el conjunto $S_2$
         			
         			\item Corre $V$ con entrada $\langle \langle s_1,s_2 \rangle ,c\rangle$ y si $V$ acepta $N$ devuelve true, en otro caso devuelve false.
         			
         		\end{enumerate}
         		
                Como dimos una MT no determinista que decide a $A$ en tiempo
                $O(n^2)$, llegamos a que $A\in NP$.
         	
            \end{proof}
         	
         \end{itemize}
        
         % Demostración 4.3
         \item $\{\langle G, G' \rangle$ $|$ $G$ y $G'$ son gráficas simples
         isomorfas\}, donde $G$ y $G'$ son isomorfas si existe una biyección
         $f: V(G) \rightarrow V(G')$ tal que $(v_1, v_2) \in E(G)$ 
         $\Longleftrightarrow (f(v_1), f(v_2)) \in E(G')$.
         
         \begin{itemize}
         	
             \item Usando definición de certificado.
         	
            \begin{proof}
         		Sea $V$ una MT total con entrada $\langle n,c \rangle$, donde $n=\langle G_1,G_2 \rangle$, siendo $G_1$ y $G_2$ gráficas, además $c=f:V(G_1)\to V(G2)$. 
         		
         		$V$ hace lo siguiente:
         		
                \begin{enumerate}[1)]
         			\item Para cada vertice $v$ en $G_1$ aplicar $f(v)$.
         			
         			\item Cada vez que se aplica $f(v)$ se reviza si los vecinos de $v$ son iguales a los vecinos de $f(v)$, si es falso que los vecinos de ambos vértices son iguales, se devuelve false, en otro caso se siguen revizando los vértices de $G_2$
         			
         			\item Al terminar de iterar sobre los vértices de $G_1$ se devuelve true.
         		
                \end{enumerate}
         		
                Como se hizo la certificación en tiempo $O(n)$, llegamos a que
                $A\in NP$. 
         	
            \end{proof}
         	
         	\newpage
         	\item Usando definición de MT no determinista
         	
            \begin{proof}
         		Sea $N$ una MT no determinista con entrada $\langle G_1,G_2 \rangle$.\\
         		
         		$N$ hace los siguiente:
         		
                \begin{enumerate}[1)]
         			\item Se definen \\
         				$A=$ el conjunto de todos los vértices de $G_1$\\
         				$B=$ el conjunto de todos los vértices de $G_2$\\
         			\item Genera al azar una función $f:A\to B$
         			
         			\item Ejecuta $V$ con entrada $\langle \langle G_1,G_2\rangle ,f\rangle$, si v devuelve true, entonces $N$ devuelve true, en otro caso devuelve false.
         		
                \end{enumerate}
         		
                Como dimos una MT no determinista que decide a $A$ en tiempo
                $O(n)$, llegamos a que $A\in NP$.
                
         	\end{proof}
         	
         \end{itemize}
         
     \end{itemize}
     
     % Ejercicio 5. 
     \item Demuestra que si $A \leq_P B$ y $C$ es cualquier lenguaje finito,
     entonces $A \cup C \leq_P B \cup C$.
     
     \begin{proof}
     	Si $w\in A \cup C$, entonces tenemos dos casos
     		
        \begin{itemize}
            
            \item Si $w\in A$, entonces se aplica $f(w)$ $\Leftrightarrow $
            $f(w)\in B \cup C$
     			
            \item Si $w\in C$, entonces se aplica la identidad $i(w)=w 
            \Leftrightarrow $ $w\in C \Leftrightarrow $ $w\in B\cup C$
        
        \end{itemize}
     	
        En ambos casos llegamos a los dos lados de la reducción por medio de
        las funciones computables $f$ e $i$, por lo tanto $A \cup C \leq_P B
        \cup C$
        
     \end{proof}
     
     % Ejercicio 6.
     \item Muestra que $SAT \leq_P 3-SAT$.
     	
     \begin{proof}
         Se dará el algoritmo para pasar de una formula en $SAT$ a una formula
         equivalente en $3-SAT$\\
     		
         Sea $C_1,C_2,...,C_n$ una formula en $SAT$, entonces revisamos cada
         cláusula de la siguiente manera:
         
         \begin{itemize}
             
             \item Si $C_i$ contiene tres variables, entonces no la modificamos.
     			
             \item Si $C_i=z$ está compuesta por una sola variable, entonces
             agregamos dos nuevas variables $x$ y $y$ y se cambia $C_i$ por
             $(z\vee x \vee y ), (z\vee x \vee \neg y ),(z\vee \neg x \vee y
             ),(z\vee \neg x \vee \neg y )$
     			
             \item Si $C_i=z\vee w$, entonces se añade una $x$ y se cambia
             $C_i$ por $(z\vee w \vee x),(z\vee w \vee \neg x)$   
     			
             \item Si $C_i=z_1 \vee z_2 \vee ... \vee z_k$ con $k>3$ variables,
             entonces se añade $k-3$  nuevas variables  $x_1,x_2,...,k_{k-3}$ y
             se cambia $C_i$ por $$(z_1\vee z_2 \vee x_1),$$
             
             $$(\neg x_1\vee z_3 \vee  x_2)$$
             $$(\neg x_2\vee z_4 \vee  x_3)$$
             $$.$$
             $$.$$
             $$.$$
             $$\neg x_{k-4}\vee z_{k-2} \vee x_{k-3}$$
             $$\neg x_{k-3}\vee z_{k-1} \vee x_{k-k}$$
      		
         \end{itemize}
     		
     	\end{proof}
     
     \newpage   
     % Ejercicio 7.
     \item Demuestra que los lenguajes $\varnothing$ y $\Sigma^{*}$
     están en $NP$ pero no son $NP-$completos.
		
     \begin{itemize}
        \item P.d. el lenguaje $\varnothing \in $ $NP$ y no es $NP-$completo.

        \begin{proof}
            Sea $M$ una MT total con entrada $\langle w \rangle$\\
				
            $M$ se devuelve true en caso de que $|w|=0$, en otro caso devuelve
            false.\\
				
            La complejidad de $M$ es $O(1)$ por lo que el lenguaje $\varnothing$
            está en $P$ y como $P \subseteq NP$ llegamos a que el lenguaje 
            vacío está en $NP$\\
				
            Ahora para demostrar que $\varnothing$ no es $NP-$Completo se hará
            la demostración por reducción al absurdo, así que supongamos que el
            lenguaje $\varnothing$ es $NP-$Completo.\\
				
            Como el lenguaje $\varnothing$ es $NP-$ Completo $\Rightarrow  
            \varnothing \leq_P A$, para algún problema $A$ $NP-$Completo, pero
            como no hay ninguna cadena en el lenguaje $\varnothing$, entonces 
            ninguna cadena está en el lenguaje $\varnothing$, por lo tanto no 
            se puede reducir el lenguaje $\varnothing$ a un problema
            $NP-$Completo. 
 				
        \end{proof}
        
        \item P.d. el lenguaje $\Sigma^{*} \in $ $NP$ y no es $NP-$completo.
        
        \begin{proof}
            Sea $M$ una MT total con entrada $\langle w \rangle$\\
				
            $M$ se devuelve true en caso de que $|w|>0$, en otro caso devuelve
            false.\\
				
            La complejidad de $M$ es $O(1)$ por lo que el lenguaje $\Sigma^{*}$
            está en $P$ y como $P \subseteq NP$ llegamos a que el lenguaje
            vacío está en $NP$.\\
				
            Ahora para demostrar que $\Sigma^{*}$ no es $NP-$Completo se hará
            la demostración por reducción al absurdo, así que supongamos que el
            lenguaje $\Sigma^{*}$ es $NP-$Completo.\\
				
            Como el lenguaje $\Sigma^{*}$ es $NP-$ Completo $\Rightarrow
            \Sigma^{*} \leq_P A$, para algún problema $A$ $NP-$Completo, pero
            como es válido que cualquier cadena esté en el lenguaje
            $\Sigma^{*}$, entonces no se puede dar ninguna cadena que no esté
            en el lenguaje $\Sigma^{*}$, por lo tanto no se puede reducir el 
            lenguaje $\Sigma^{*}$ a un problema $NP-$Completo.
				
        \end{proof}
        
		\end{itemize}
    
    \end{enumerate}

\end{document}


