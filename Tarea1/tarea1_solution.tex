\documentclass[letterpaper,11pt]{article}

% Soporte para los acentos.
\usepackage[utf8]{inputenc}
\usepackage[T1]{fontenc}    
% Idioma español.
\usepackage[spanish,mexico, es-tabla]{babel}
% Soporte de símbolos adicionales (matemáticas)
\usepackage{multirow}
\usepackage{amsmath}		
\usepackage{amssymb}		
\usepackage{amsthm}
\usepackage{amsfonts}
\usepackage{latexsym}
% Modificamos los márgenes del documento.
\usepackage[lmargin=2cm,rmargin=2cm,top=2cm,bottom=2cm]{geometry}
% Soporte para dibujar con Tikz.
\usepackage{tikz}
\usetikzlibrary{automata, positioning, arrows}

% Información para el título
\title{Autómatas y Lenguajes Formales\\ Tarea 1}
\author{Rubí Rojas Tania Michelle}
\date{13 de septiembre de 2018}

\begin{document}
\maketitle

\begin{enumerate}

% EJERCICIO 1.
\item Si $A$ es un lenguaje, demuestra que $(A^{*})^{*} = A^{*}$.
  \begin{proof}[Demostración:]
    Supongamos que A es un lenguaje.
    \begin{align*}
    x \in (A^{*})^{*}  &\Leftrightarrow  
x \in \bigcup\limits_{i=0}^{\infty}(\bigcup\limits_{j=0}^{\infty}  A^{j})^{i} \\
&\Leftrightarrow \exists k, m_1, ..., m_k \in \mathbb{N} : x = x_1, ..., x_k ;  
 x_1 \in A^{m_1}, ..., x_k \in A^{m_k} \\
 &\Leftrightarrow \exists r \in \mathbb{N} : x \in A^{r}, r = m_1 + ... + m_k \\
 &\Leftrightarrow x \in \bigcup\limits_{i=0}^{\infty} A^{i} \\
 &\Leftrightarrow x \in A^{*}
    \end{align*}
  \end{proof}

  % EJERCICIO 2.
\item Si $A$, $B$ y $C$ son lenguajes, prueba que $A(B \cup C) = AB \cup AC$.
  \begin{proof}[Demostración:]
    Supongamos que $A$, $B$ y $C$ son lenguajes.
    \begin{itemize}
      % Primera contención de la demostración.
    \item[$\subseteq)$] Sea $x \in A(B \cup C).$ Entonces $x = ak$ para las
      cadenas $a \in A$ y $k \in B \cup C$, por lo que $a \in A$ y $(k \in B$ ó
      $k \in C)$.
      % Casos de la contención.
      \begin{itemize}
      \item[i)] $a \in A$ y $k \in B$. Entonces $ak \in AB$ y por la
        definición de unión concluimos que $x = ak \in AB \cup AC$.
      \item[ii)] $a \in A$ y $k \in C$. Entonces $ak \in AC$ y por la
        definición de unión concluimos que $x = ak \in AB \cup AC$.
      \end{itemize}
      % Segunda contención de la demostración.
    \item[$\supseteq)$] Sea $x \in AB \cup AC$. Entonces $x \in AB$ ó$x \in AC$.
      % Casos de la contención.
      \begin{itemize}
      \item[i)] $x \in AB$. Entonces $x = ab$ para las cadenas  $a \in A$ y
        $b \in B$. Por la definición de la unión tenemos que $b \in B \cup C$.
        Por lo tanto, $x = ab \in A(B \cup C)$.
      \item[ii)] $x \in AC$. Entonces $x = ac$ para las cadenas $a \in A$ y
        $c \in C$. Por la definición de la unión tenemos que $c \in B \cup C$.
        Por lo tanto, $x = ac \in A(B \cup C)$.
    \end{itemize}
    \end{itemize}
    \end{proof}
  
  ¿Es cierto que $A(B \cap C) = AB \cap AC$? Argumenta su respuesta.
  \textsc{\\Solución:\\} No es cierto. Para justificarlo mostraré un
  contraejemplo:
  Sean $A=\{x$, $\epsilon\}$, $B=\{\epsilon\}$ y $C=\{x\}$. Entonces 
  $AB = \{x$, $\epsilon\}$, $AC = \{x^{2}$, $x\}$ y $B \cap C = \varnothing$, 
  de donde $A(B \cap C) = \varnothing$ $\neq$ $\{x\} = AB \cap AC$.
  
  % EJERCICIO 3.
\item Muestra que los lenguajes regulares son cerrados bajo intersección
  utilizando autómatas finitos deterministas.
  \begin{proof}[Demostración:] 
  Supongamos que $A$ y $B$ son lenguajes regulares. Entonces existen
  autómatas finitos deterministas $M_A = (Q_A, \sum , \delta_A, q_A, F_A)$
  y $M_A = (Q_B, \sum , \delta_B, q_B, F_B)$ tales que reconocen a A y B, 
  respectivamente. Ahora, construimos un autómata finito determinista
  $M = (Q, \sum , \delta, q_0, F)$ tal que definido como
  \begin{itemize}
   \item[i)] $Q = Q_A \times Q_B = \{(q_1, q_2) : q_1 \in Q_A \wedge q_2 \in 
                        Q_B\}$
   \item[ii)] $\sum =$ es el mismo alfabeto que en $M_A$ y $M_B$.
   \item[iii)] $\delta ((q_1, q_2), x) = (\delta_A(q_1, x), \delta_B(q_2, x))$
   \item[iv)] $q_0 = (q_1, q_2)$
   \item[v)] $F = F_A \times F_B =\{(q_1, q_2) : q_1 \in F_A \wedge q_1 \in 
                          F_B\}$      
  \end{itemize}
  reconoce a $A \cap B$.\\
  Entonces sólo queda probar que el lenguaje aceptado por $M$ es igual a 
  $A \cap B$.\\ \\
  P.D. $L(M) = A \cap B$.
  \begin{itemize}
    %Primera contención de la demostración.
   \item[$\subseteq )$] Sea $x \in L(M)$.  Entonces $\delta^{*} (q_0, x) \in F$
   de donde $\delta^{*} ((q_1, q_2), x)$ por la definición de $q_0$. Luego, 
   por la definición de $\delta^{*}$ tenemos que $( \delta^{*} (q_1, x), 
   \delta^{*} (q_2, x)) \in F$ por lo que $\delta_A^{*} (q_1, x) \in F_A$ y 
   $\delta_B^{*} (q_2, x) \in F_B$. Finalmente, $x \in A$ y $x \in B$. Por 
   lo tanto, $x \in A \cap B$.
   %Segunda contención de la demostración.
   \item[$\supseteq )$] Sea $x \in A \cap B$. Entonces $x \in A$ y $x \in B$, de
   donde $\delta_A^{*} (q_A, x) \in F_A$ y $\delta_B^{*} (q_B, x) \in F_B$.
   Luego, por la definición de $F$ tenemos que  $( \delta^{*} (q_1, x), 
   \delta^{*} (q_2, x)) \in F$ por lo que $\delta^{*} ((q_A, q_B), x) \in F$
   por la definición de $\delta^{*}$. Finalmente, $\delta^{*} (q_0, x) \in F$
   por la definición de $q_0$. Por lo tanto, $x \in L(M)$.
  \end{itemize}
  \end{proof}

  % EJERCICIO 4.
\item Demuestra que cualquier conjunto finito de cadenas es un lenguaje regular.
  \begin{proof}[Demostración:]
    Si $A$ es el conjunto vacío entonces está definido por la expresión regular
    $\emptyset$, por lo que $A$ es un lenguaje regular. Si $A$ es algún lenguaje
    finito compuesto por las cadenas $s_{1}$, $s_{2}$, $...$, $s_{n}$ para algún
    entero positivo $n$, entonces está definido por la expresión regular
    $s_{1} + s_{2} + ... + s_{n}$, por lo que $A$ es un lenguaje 
regular.
    Por lo tanto, cualquier conjunto finito de cadenas es un lenguaje regular.\\
  \end{proof}

  % EJERCICIO 5.
  \item Demuestra que para todas las cadenas $x$, $y \in \sum^{*}$ se cumple
  $\delta^{*}(q$, $xy) = \delta^{*}(\delta^{*}(q$, $x)$, $y)$.
  \begin{proof}[Demostración:]
    Inducción sobre $|y|$.\\
    \textit{Base de inducción: } Veamos que se cumple para $|y| = 0$.\\
    Si $|y| = 0$ entonces $y = \epsilon$. Luego
    \begin{align*}
    \delta^{*}(q, xy) &= \delta^{*}(q, x \epsilon) && \text{definición de $y$}\\
                      &= \delta^{*}(q, x) && \text{definición de $\epsilon$}\\
                      &= \delta^{*}(q, x \epsilon) && \text{definición de $\epsilon$}\\
                      &= \delta^{*}(\delta^{*}(q,x), \epsilon) && \text{definición de $\delta^{*}$}\\
                      &=  \delta^{*}(\delta^{*}(q,x), y) && \text{definición de $y$}
    \end{align*}
    Por lo tanto, se cumple para $|y| = 0$.\\ 
    
    \textit{Hipótesis de inducción: } Supongamos que se cumple para
    $0 \leq |y| \leq n$, es decir, que se cumple para
    $\delta^{*}(q$, $xy) = \delta^{*}(\delta^{*}(q$, $x)$, $y)$. \\

    \textit{Paso inductivo:} Queremos probar que se cumple para
    $|y| = n + 1$.\\ Sea $y = za$ con $z \in \sum^{*}$, $a \in \sum$ y
    $|z| = n$. Entonces
    \begin{align*}
    \delta^{*}(q, xy) &= \delta^{*}(q, xza) && \text{definición de $y$}\\
                      &= \delta(\delta^{*}(q, xz), a) && \text{definición de $\delta^{*}$}\\
                      &= \delta(\delta^{*}(\delta^{*}(q, x), z), a) && \text{hipótesis de inducción}\\
                      &= \delta^{*}(\delta^{*}(q,x), za) && \text{definición de $\delta^{*}$}\\
                      &=  \delta^{*}(\delta^{*}(q,x), y) && \text{definición de $y$}
    \end{align*}
  \end{proof}
  
  % EJERCICIO 6.
\item Para cada uno de los siguientes lenguajes diseña un \textsc{AFD} que lo
  reconozca:
  \begin{itemize}
 % \setlength\itemsep{9em}
    % Lenguaje 1.
  \item \{$w$ $|$ $w$ contiene al menos tres $a$ y un número par de $b$.\}
    \textit{\\Solución:}
    \begin{figure}[ht]
    \centering
    	\begin{tikzpicture}[->,>=stealth',shorten >=1pt,auto,node 
distance=2.8cm,
    	semithick]
    	% Asignamos los estados del autómata.
    	\node[state,  initial] (q0) {$q_0$};
    	\node[state, right of=q0] (q1) {$q_1$};
    	\node[state, below of=q0] (q2) {$q_2$};
    	\node[state, right of=q2] (q3) {$q_3$};
    	\node[state, below of=q2] (q4) {$q_4$};
    	\node[state, right of=q4] (q5) {$q_5$};
    	\node[state, accepting, below of=q4] (q6) {$q_6$};
    	\node[state, right of=q6] (q7) {$q_7$};
    	% Dibujamos las transiciones del autómata.
    	\draw (q0) edge[bend left] node{b} (q1)
    	(q0) edge[above left] node{a} (q2)
    	(q1) edge[bend left] node{b} (q0)
    	(q1) edge[above left] node{a} (q3)
    	(q2) edge[bend left] node{b} (q3)
    	(q2) edge[above left] node{a} (q4)
    	(q3) edge[bend left] node{b} (q2)
    	(q3) edge[above left] node{a} (q5)
    	(q4) edge[bend left] node{b} (q5)
    	(q4) edge[above left] node{a} (q6)
    	(q5) edge[bend left] node{b} (q4)
    	(q5) edge[above left] node{a} (q7)
    	(q6) edge[bend left] node{b} (q7)
    	(q6) edge[loop below] node{a} (q6)
    	(q7) edge[bend left] node{b} (q6)
    	(q7) edge[loop below] node{a} (q7);
    	\end{tikzpicture}
    \end{figure}
    \newpage
    
    % Lenguaje 2.
  \item \{ $w$ $|$ $w$ contiene exactamente tres $a$ y una $b$\}
    \textit{\\Solución:}
     \begin{figure}[ht]
    \centering
    	\begin{tikzpicture}[->,>=stealth',shorten >=1pt,auto,node 
distance=2.8cm,
    	semithick]
    	% Asignamos los estados del autómata.
    	\node[state,  initial] (q0) {$q_0$};
    	\node[state, right of=q0] (q1) {$q_1$};
    	\node[state, right of=q1] (q2) {$q_2$};
    	\node[state, right of=q2] (q3) {$q_3$};
    	\node[state, below  of=q0] (q4) {$q_4$};
    	\node[state, right of=q4] (q5) {$q_5$};
    	\node[state, right of=q5] (q6) {$q_6$};
    	\node[state, accepting, right of=q6] (q7) {$q_7$};
    	% Dibujamos las transiciones del autómata.
    	\draw (q0) edge[above left] node{a} (q1)
    	(q0) edge[above left] node{b} (q4)
    	(q1) edge[above left] node{a} (q2)
    	(q1) edge[above left] node{b} (q5)
    	(q2) edge[above left] node{a} (q3)
    	(q2) edge[above left] node{b} (q6)
    	(q3) edge[above left] node{b} (q7)
    	(q4) edge[above left] node{a} (q5)
    	(q4) edge[loop below] node{b} (q4)
    	(q5) edge[above left] node{a} (q6)
    	(q5) edge[loop below] node{b} (q5)
    	(q6) edge[above left] node{a} (q7)
    	(q6) edge[loop below] node{b} (q6)
    	(q7) edge[loop below] node{b} (q7);
    	\end{tikzpicture}
    \end{figure}

    % Lenguaje 3.
  \item \{$w$ $|$ $w$ es una cadena de longitud par y contiene un número par
    de $b$\}
    \textit{\\Solución:}
    \begin{figure}[ht]
    	\centering
    	\begin{tikzpicture}[->,>=stealth',shorten >=1pt,auto,node distance=2.8cm,
    	semithick]
    	% Asignamos los estados del autómata.
    	\node[state, accepting, initial] (q0) {$q_0$};
    	\node[state, below of=q0] (q1) {$q_1$};
    	\node[state, right of=q1] (q2) {$q_2$};
    	\node[state, right of=q0] (q3) {$q_3$};
    	% Dibujamos las transiciones del autómata.
    	\draw (q0) edge[bend left] node{b} (q1)
    	(q0) edge[above] node{a} (q3)
    	(q1) edge[bend left] node{b} (q0)
    	(q1) edge[above] node{a} (q2)
    	(q2) edge[bend left] node{a} (q1)
    	(q2) edge[bend left] node{b} (q3)
    	(q3) edge[bend left] node{a} (q0)
    	(q3) edge[bend left] node{b} (q2);
    	\end{tikzpicture}
    \end{figure}
    
    % Lenguaje 4.
    \item \{$w$ $|$ $w$ empieza con $a$ y termina con dos $b$\}
    \textit{\\Solución:}
    \begin{figure}[ht]
    	\centering
    	\begin{tikzpicture}[->,>=stealth',shorten >=1pt,auto,node distance=2.8cm,
          semithick]
        % Asignamos los estados del autómata.
        \node[state, initial] (q0) {$q_0$};
        \node[state, right of=q0] (q1) {$q_1$};
        \node[state,  above right of=q1] (q2) {$q_2$};
        \node[state,  below right of=q1] (q3) {$q_3$};
        \node[state, accepting, right of=q3] (q4) {$q_4$};
        % Dibujamos las transiciones del autómata.
        \draw (q0) edge[above] node{a} (q1)
        (q1) edge[above] node{a} (q2)
        (q1) edge[above] node{b} (q3)
        (q2) edge[loop above] node{a} (q2)
        (q2) edge[bend left] node{b} (q3)
        (q3) edge[bend left] node{a} (q2)
        (q3) edge[above] node{b} (q4)
        (q4) edge[bend right] node{a} (q2)
        (q4) edge[loop below] node{b} (q4);
    \end{tikzpicture}
\end{figure}
  \end{itemize}
  
  \newpage
  % EJERCICIO 7.
\item Para cada uno de los siguientes lenguajes diseña un \textsc{AFN}
  (puede o no tener transiciones-$\epsilon$) que lo reconozca:
  \begin{itemize}

    % Lenguaje 1.
  \item \{$w$ $|$ $w$ es una cadena que contiene una $a$ en la primer posición
    o una $b$ en la penúltima posición\}
    \textit{\\Solución:}
    \begin{figure}[ht]
      \centering
      \begin{tikzpicture}[->,>=stealth',shorten >=1pt,auto,node distance=2.8cm,
          semithick]
        % Asignamos los estados del autómata.
        \node[state, initial](q){$q$};
        \node[state, below of=q] (q0) {$q_0$};
        \node[state, accepting, right of=q0] (q1) {$q_1$};
        \node[state, right of=q1] (r0) {$r_0$};
        \node[state, right of=r0](r1) {$r_1$};
        \node[state, accepting, right of=r1](r2) {$r_2$};
        % Dibujamos las transiciones del autómata.
        \draw (q) edge[bend right] node{$\epsilon$} (q0)
        (q) edge[bend left] node{$\epsilon$} (r0)
        (q0) edge[above] node{a} (q1)
        (q1) edge[loop above] node{a,b} (q1)
        (r0) edge[loop below] node{a,b} (r0)
        (r0) edge[above] node{b} (r1)
        (r1) edge[above] node{a,b} (r2);
    \end{tikzpicture}
    \end{figure}
    
    % Lenguaje 2.
  \item\{$xy$ $|$ $x$ contiene la subcadena $aa$ e $y$ contiene una $b$\}
    \textit{\\Solución:}
    \begin{figure}[ht]
    	\centering
    	\begin{tikzpicture}[->,>=stealth',shorten >=1pt,auto,node distance=2.8cm,
    	semithick]
    	% Asignamos los estados del autómata.
    	\node[state, initial] (q0) {$q_0$};
    	\node[state, right of=q0] (q1) {$q_1$};
    	\node[state, right of=q1] (q2) {$q_2$};
    	\node[state, below of=q2](q3) {$q_3$};
    	\node[state, accepting, right of=q3](q4) {$q_4$};
    	% Dibujamos las transiciones del autómata.
    	\draw (q0) edge[above] node{a} (q1)
    	(q0) edge[loop above] node{b} (q0)
    	(q1) edge[above] node{a} (q2)
    	(q1) edge[bend left] node{b} (q0)
    	(q2) edge[loop above] node{a,b} (q2)
    	(q2) edge[bend right] node{$\epsilon$} (q3)
    	(q3) edge[above] node{b} (q4)
    	(q3) edge[loop below] node{a} (q3)
    	(q4) edge[loop above] node{a,b} (q4);
    	\end{tikzpicture}
    \end{figure}
  \end{itemize}

  % EJERCICIO 8.
\item Si $A$ es un lenguaje regular, demuestra que $A^{R} = \{w^{R} | w \in A\}$.
  \begin{proof}[Demostración:]
    Supongamos que $A$ es un lenguaje regular. Sea $E$ una expresión 
    regular para $A$. Vamos a mostrar que hay otra expresión regular $E^{R}$
    tal que $A(E^{R}) = (A(E))^{R}$, es decir, el lenguaje de $E^{R}$ es la
    reversa del lenguaje $E$.\\ \\
    Inducción sobre $E$.\\
    \textit{Base de inducción:} \\ 
    Si $E$ es $\epsilon$, $\varnothing$ o un símbolo $a$, entonces 
    $E^{R} = E$.\\
    \textit{Paso inductivo: } \\ 
    Hay tres casos, dependiendo de la forma de $E$.
    \begin{itemize}
     \item[i)] $E = F + G$. La reversa de la unión de dos lenguajes se obtiene
     calculando la reversa de los dos lenguajes y tomando la unión de estos
     lenguajes. Por lo tanto, $E^{R} = F^{R} + G^{R}$.
     \item[ii)] $E = FG$.  En general, si una cadena $w \in A(E)$ es la 
     concadenación de $w_1 \in A(F)$ y $w_2 \in A(G)$, entonces 
     $w^{R} = w_2^{R} w_1^{R}$. Por lo tanto $E^{R} = G^{R} F^{R}$.
     \item[iii)] $E = F^{*}$. Cada cadena $w \in A(E)$ puede ser escrita como
     $w_1 w_2 ... w_n$, donde cada $w_i \in A(F)$. Pero 
     $w^{R} = w_n^{R} ... w_2^{R} w_1^{R}$ y cada $w_i^{R} \in A(E^{R})$,
     así $w^{R} \in A((F^{R})^{*})$.\\
     Inversamente, cada cadena en $A((F^{R})^{*})$ es de la forma 
     $w_1 w_2 ... w_n$ donde cada $w_i$ es la reversa de una cadena en 
     $A(F)$. La reversa de esta cadena está en $A(F)$ que está en $A(E)$.
     Por lo tanto, gracias a las observaciones anteriores tenemos que
     $E^{R} = (F^{R})^{*}$.
    \end{itemize}
    \end{proof}
  
    % EJERCICIO 9.
\item Elige dos autómatas del ejercicio 6 y minimízalos. También elige un
autómata del ejercicio 7 y minimízalo.
    \textit{\\Solución:\\}
    % Autómatas del ejercicio 6.
    Del ejercicio 6 escogí los últimos dos autómatas. La minimización la
    presento a continuación:
    \begin{itemize}
     \item[i)] \{$w$ $|$ $w$ es una cadena de longitud par y contiene un número
    par de $b$\} \\
    Sea $X = \{q_0\}$ el conjunto de estados finales y $Y = \{q_1, q_2, q_3\}$
    el conjunto de estados restantes. Entonces obtenemos las siguientes 
    tablas:\\ \\
    \begin{center}
    \begin{tabular}{||l | l | l||}
    \hline
    \hline
     & a & b \\
     \hline
     $q_0$ & Y & Y\\
     \hline
     \end{tabular}
     \begin{tabular}{||l | l | l||}
     
    \hline
    \hline
     & a & b \\
     \hline
     $q_1$ & Y & X\\
     \hline
     $q_2$ & Y & Y \\
     \hline
     $q_3$ & X & Y\\
     \hline
     \end{tabular}
     \end{center}
     .\\ \\
     Como ningún estado tiene caractetísticas en común con algún otro de su
     conjunto entonces el autómata ya no se puede minimizar más. Por lo
     tanto, el autómata minimizado es el autómata original.
    
     \item[ii)] \{$w$ $|$ $w$ empieza con $a$ y termina con dos $b$\} \\
     Sea $X = \{q_4\}$ el conjunto de estados finales y 
     $Y = \{q_0, q_1, q_2, q_3\}$ el conjunto de estados restantes. Entonces
     obtenemos las siguientes tablas:\\ \\
     \begin{center}
    \begin{tabular}{||l | l | l||}
    \hline
    \hline
     & a & b \\
     \hline
     $q_4$ & Y & X\\
     \hline
     \end{tabular}
     \begin{tabular}{||l | l | l||}
    \hline
    \hline
     & a & b \\
     \hline
     $q_0$ & Y & \\
     \hline
     $q_1$ & Y & Y \\
     \hline
     $q_2$ & Y & Y\\
     \hline
     $q_3$ & Y & X\\
     \hline
     \end{tabular}
     \end{center}
     . \\ \\
     Como $q_1$ y $q_2$ tienen caractetísticas en común, entonces repetimos
     el proceso anterior, pero ahora con los siguientes conjuntos:
     $X = \{q_4\}$, $Y = \{q_0, q_2\}$ y $Z = \{q_1, q_2\}$. Entonces obtenemos
     las siguientes tablas:\\ \\
     \begin{center}
    \begin{tabular}{||l | l | l||}
    \hline
    \hline
     & a & b \\
     \hline
     $q_4$ & Z & X\\
     \hline
     \end{tabular}
     \begin{tabular}{||l | l | l||}
    \hline
    \hline
     & a & b \\
     \hline
     $q_0$ & Z & \\
     \hline
     $q_3$ & Z & X \\
     \hline
     \end{tabular}
     \begin{tabular}{||l | l | l||}
    \hline
    \hline
     & a & b \\
     \hline
     $q_1$ & Z & Y \\
     \hline
     $q_2$ & Z & Y\\
     \hline
     \end{tabular}
     \end{center}
     .\\ \\ \\
     Por lo tanto, el autómata minimizado sería de la siguiente forma:
     \begin{figure}[ht]
    	\centering
    	\begin{tikzpicture}[->,>=stealth',shorten >=1pt,auto,node 
distance=2.8cm,
    	semithick]
    	% Asignamos los estados del autómata.
    	\node[state, initial] (q0) {$q_0$};
    	\node[state, right of=q0] (q1) {$q_1$};
    	\node[state, below of=q1] (q2) {$q_2$};
    	\node[state, right of=q2](q3) {$q_3$};
    	% Dibujamos las transiciones del autómata.
    	\draw (q0) edge[above] node{a} (q1)
    	(q1) edge[loop above] node{a} (q0)
    	(q1) edge[bend left] node{b} (q2)
    	(q2) edge[bend left] node{a} (q1)
    	(q2) edge[above] node{b} (q3)
    	(q3) edge[loop below] node{b} (q3)
    	(q3) edge[bend right] node{a} (q1);
    	\end{tikzpicture}
    \end{figure}
    \end{itemize}
    Del ejercicio 7 escogí el último autómata, y la minimización se presenta
    a continuación:
    % Autómata del ejercico 7.
    \begin{itemize}
     \item[i)] \{$xy$ $|$ $x$ contiene la subcadena $aa$ e $y$ contiene 
     una $b$\} \\
     Primero hallamos las clausuras respecto a $\epsilon$.
     \begin{itemize}
      \item $C-\epsilon (0) = \{q_0\}$
      \item $C-\epsilon (1) = \{q_1\}$
      \item $C-\epsilon (2) = \{q_2\} \cup \{q_3\}$
      \item $C-\epsilon (3) = \{q_3\}$
      \item $C-\epsilon (4) = \{q_4\}$
     \end{itemize}
     Luego, elaboramos la tabla de transiciones del $AFN-\epsilon$.
     \begin{center}
      \begin{tabular}{||l | l | l | l||}
    \hline
    \hline
     Q & $\epsilon$ & a & b \\
     \hline
     $q_0$ & $\varnothing$ & $q_1$ & $q_0$ \\
     \hline
     $q_1$ & $\varnothing$ & $q_2$ & $q_0$ \\
     \hline
     $q_2$ & $q_3$ & $q_2$ & $q_2$\\
     \hline
     $q_3$ & $\varnothing$ & $q_3$ & $q_4$\\
     \hline
     $q_4$ & $\varnothing$ & $q_4$ & $q_4$\\
     \hline
     \end{tabular}
     \end{center}
     .\\ \\
     Ahora, hacemos una tabla de conversion para eliminar las 
     transiciones-$\epsilon$.
     \begin{center}
      \begin{tabular}{||l  | l | l||}
    \hline
    \hline
     Q  & a & b \\
     \hline
     $\{q_0\}$  & $\{q_1\}$ & $\{q_0\}$ \\
     \hline
     $\{q_1\}$ & $\{q_2\} \cup \{q_3\}$ & $\{q_0\}$ \\
     \hline
     $\{q_2\} \cup \{q_3\}$  & $\{q_2\} \cup \{q_3\}$  & 
     $\{q_2\} \cup \{q_4\} \cup \{q_3\}$ \\
     \hline
     $\{q_2\} \cup \{q_4\} \cup \{q_3\}$  & 
     $\{q_2\} \cup \{q_4\} \cup \{q_3\}$ &
     $\{q_2\} \cup \{q_4\} \cup \{q_3\}$ \\
     \hline
     \end{tabular}
     \end{center}
     \newpage
     Así, tenemos que $A = \{q_0\}$, $B = \{q_1\}$, 
     $C = \{ \{q_2\} \cup \{q_3\} \}$ y
     $D = \{ \{q_2\} \cup \{q_4\} \cup \{q_3\} \}$.
     De donde obtenemos la siguiente tabla: \\
     \begin{center}
      \begin{tabular}{||l  | l | l||}
    \hline
    \hline
     Q  & a & b \\
     \hline
     A  & B & A\\
     \hline
     B & C & A\\
     \hline
     C  & C & D\\
     \hline
     D  & D & D\\
     \hline
     \end{tabular}
     \end{center}
     .\\ \\
     Por lo tanto, el autómata minimizado es de la siguiente forma:
     \begin{figure}[ht]
    	\centering
    	\begin{tikzpicture}[->,>=stealth',shorten >=1pt,auto,node 
distance=2.8cm,
    	semithick]
    	% Asignamos los estados del autómata.
    	\node[state, initial] (A) {$A$};
    	\node[state, right of=A] (B) {$B$};
    	\node[state, right of=B] (C) {$C$};
    	\node[state, right of=C](D) {$D$};
    	% Dibujamos las transiciones del autómata.
    	\draw (A) edge[bend left] node{a} (B)
    	(A) edge[loop above] node{b} (A)
    	(B) edge[above] node{a} (C)
    	(B) edge[bend left] node{b} (A)
    	(C) edge[loop above] node{a} (C)
    	(C) edge[above] node{b} (D)
    	(D) edge[loop above] node{a,b} (D);
    	\end{tikzpicture}
    \end{figure}
    \end{itemize}


    % EJERCICIO 10.    
    \item Expresiones regulares:
    \begin{itemize}
    % Ejercicio 10.1
    \item Da una expresión regular que genere cada uno de los lenguajes
    del ejercicio 6.
    \begin{itemize}
    \setlength\itemsep{2em}
    % Lenguaje 1.
    \item \{$w$ $|$ $w$ contiene al menos tres $a$ y un número par de $b$.\}
    \textit{\\Solución:}
    $(a+b)^{*}a(a+b)^{*}a(a+b)^{*}a(a+b)^{*}(bb)^{*}(a+b)^{*} + $\\
    $(a+b)^{*}a(a+b)^{*}a(a+b)^{*}(bb){*}(a+b)^{*}a(a+b)^{*} + $\\
    $(a+b)^{*}a(a+b)^{*}(bb)^{*}(a+b)^{*}a(a+b)^{*}a(a+b)^{*} + $\\
    $(a+b)^{*}(bb)^{*}(a+b)^{*}a(a+b)^{*}a(a+b)^{*}a(a+b)^{*}$.
    
    % Lenguaje 2.
    \item \{$w$ $|$ $w$ contiene exactamente tres $a$ y una $b$\}
    \textit{\\Solución:} $(aaab)+(aaba)+(abaa)+(baaa)$
    
    % Lenguaje 3.
    \item \{$w$ $|$ $w$ es una cadena de longitud par y contiene un número 
    par de $b$\}
    \textit{\\Solución:} $((bb)^{*}(a+b)(bb)^{*}(a+b)(bb)^{*})^{*}$
    
    % Lenguaje 4.
    \item \{$w$ $|$ $w$ empieza con $a$ y termina con dos $b$\}
    \textit{\\Solución:} $a(a+b)^{*}bb$
    \end{itemize}
    \newpage
    % Ejercicio 10.2
    \item Construye un autómata finito (puede ser no determinista) que reconozca
    el lenguaje generado por la expresión $(1 + 0(01^{*}0)^{*})^{*}$.
    \textit{\\Solución:}
    \begin{figure}[ht]
      \centering
      \begin{tikzpicture}[->,>=stealth',shorten >=1pt,auto,node distance=2.5cm,
          semithick]
        % Asignamos los estados del autómata.
        \node[state, accepting, initial](q0){$q_0$};
        \node[state, right of=q0] (q1) {$q_1$};
        \node[state, above of=q1] (q2) {$q_2$};
        \node[state, accepting, right of=q2] (q3) {$q_3$};
        \node[state, below  of=q1] (q4) {$q_4$};
        \node[state, right of=q4] (q5) {$q_5$};
        \node[state, accepting, right of=q5] (q6) {$q_6$};
        \node[state, right of=q6](q7) {$q_7$};
        \node[state, right of=q7](q8) {$q_8$};
        \node[state, right of=q8](q9) {$q_9$};
        \node[state, below  of=q9](q10) {$q_{10}$};
        \node[state, left of=q10](q11) {$q_{11}$};
        \node[state, left of=q11](q12) {$q_{12}$};
        \node[state, accepting, below left of=q4](q13) {$q_{13}$};
        % Dibujamos las transiciones del autómata.
        \draw (q0) edge[above] node{$\epsilon$} (q1)
        (q1) edge[above left] node{$\epsilon$} (q2)
        (q2) edge[above left] node{$1$} (q3)
        (q3) edge[bend left] node{$\epsilon$} (q1)
        (q1) edge[above left] node{$\epsilon$} (q4)
        (q4) edge[above left] node{$0$} (q5)
        (q5) edge[above left] node{$\epsilon$} (q6)
        (q6) edge[above left] node{$\epsilon$} (q7)
        (q7) edge[above left] node{$0$} (q8)
        (q8) edge[above left] node{$\epsilon$} (q9)
        (q9) edge[above left] node{$\epsilon$} (q10)
        (q10) edge[bend left] node{$1$} (q11)
        (q11) edge[bend left] node{$1$} (q10)
        (q11) edge[above left] node{$\epsilon$} (q12)
        (q12) edge[above left] node{$0$} (q13)
        (q6) edge[above left] node{$\epsilon$} (q1)
        (q9) edge[above left] node{$\epsilon$} (q12)
        (q13) edge[above right] node{$\epsilon$} (q7)
        (q13) edge[above left] node{$\epsilon$} (q1);
        \end{tikzpicture}
    \end{figure}
    
    % Ejercicio 10.3
    \item Construye un \textsc{AFN} que reconozca el lenguaje generado por
    la expresión $(a + b)^{*}aba$.
    \textit{\\Solución:}
    \begin{figure}[ht]
    	\centering
    	\begin{tikzpicture}[->,>=stealth',shorten >=1pt,auto,node 
distance=4cm,
    	semithick]
    	% Asignamos los estados del autómata.
    	\node[state, initial] (q0) {$q_0$};
    	\node[state, right of=q0] (q1) {$q_1$};
    	\node[state, right of=q1] (q2) {$q_2$};
    	\node[state, accepting,  right of=q2] (q3) {$q_3$};
    	% Dibujamos las transiciones del autómata.
    	\draw (q0) edge[above left] node{a} (q1)
    	(q0) edge[loop above] node{b} (q0)
    	(q1) edge[loop above] node{a} (q1)
    	(q1) edge[above left] node{b} (q2)
    	(q2) edge[above right] node{a} (q3)
    	(q2) edge[bend left] node{b} (q0)
    	(q3) edge[bend right] node{a} (q1)
    	(q3) edge[bend left] node{b} (q0);
    	\end{tikzpicture}
    \end{figure}

    \end{itemize}
    
    % Ejercicio 11.
    \item Usando el lema del bombeo, prueba que los siguientes lenguajes no son
    regulares.
    \begin{itemize}
    
    % Lenguaje 1.    
    \item \{$x$ $|$ $x$ es una cadena palíndroma\}
    \begin{proof}[Demostración:] Veamos que $L$ no es regular.
    \begin{itemize}
    \item[i)] Sea $k \geq 0$.
    \item[ii)] $xyz = 0^{k}1^{k}0^{k}$ con $x = \epsilon$, $y = 0^{k}$ y $z = 1^{k}0^{k}$.
    \item[iii)] Ten $u,v,w$ tales que $y = uvw$ con $v \neq \epsilon$.
    \item[iv)] Sea $i=2$. Entonces $|uv^{i}w| > k$ por lo que $xuv^{i}wz \notin L$.
    \end{itemize}
    Por lo tanto, $L$ no es un lenguaje regular.\\
    \end{proof}
    \newpage
    % Lenguaje 2.    
    \item \{$a^{2^{n}}$ $|$ $n \in \mathbb{N}$\}
    \begin{proof}[Demostración:] Veamos que $L$ no es regular.
    	\begin{itemize}
    		\item[i)] Sea $k \geq 0$.
    		\item[ii)] $xyz = a^{2^{k}}$ con $x= a^{n}$, 
    		$y= a^{m} $, $z = a^{p}$ tal que $n+m+p = 2^{k}$.
    		\item [iii)] Ten $u,v,w$ tales que $y = uvw$ con $v \neq \epsilon$.
    		\item[iv)] Sea $i=2$. Entonces $|uv^{i}w| > k$ por lo que 
$xuv^{i}wz \notin L$.
    	\end{itemize}
%     Por lo tanto, $L$ no es un lenguaje regular.\\
    \end{proof}
    
    % Lenguaje 3.
    \item \{$x$ $|$ $x$ tiene el mismo número de $a$ y $b$\}
    \begin{proof}[Demostración:] Veamos que $L$ no es un lenguaje regular.
    \begin{itemize}
    \item[i)] Sea $k \geq 0$.
    \item[ii)] $xyz = a^{k}b^{k}$ con $x = \epsilon$, $y = a^{k}$ y $z = b^{k}$.
    \item[iii)] Ten $u,v,w$ tales que $y = uvw$ con $v \neq \epsilon$.
    \item[iv)] Sea $i=z$. Entonces $|uv^{i}w| > k$ por lo que $xuv^{i}wz \notin L$.
    \end{itemize}
    Por lo tanto, $L$ no es un lenguaje regular.\\
    \end{proof}
    \end{itemize}
     
    % Ejercicio 12.
    \item Utilizando el teorema de Myhill-Nerode demuestra que cada uno de
    los lenguajes del inciso anterior no son regulares.
    \begin{itemize}
    
    % Lenguaje 1.
    \item \{$x$ $|$ $x$ es una cadena palíndroma\}
    \begin{proof}[Demostración:] Primero tomamos dos cadenas cualesquiera, 
    sean $x = a^{i}b$ y $y = a^{j}b$ con $i \neq j$. Ahora, sea $z = a^{i}$.
    Entonces $xz = a^{i}ba^{i} \in L$ pues $xz$ es palíndroma
    pero $yz = a^{j}ba^{i} \notin L$ ya que $yz$ no es palíndroma.\\
    Por lo tanto, $L$ no es un lenguaje regular.\\      
    \end{proof}
    
    % Lenguaje 2.    
    \item \{$a^{2^{n}}$ $|$ $n \in \mathbb{N}$\}
    \begin{proof}[Demostración:] Primero tomamos dos cadenas cualesquiera,
    	sean $x = a^{2^{i}+1}$ y $y = a^{2^{j}+2}$, con $i \neq j$. Ahora,
    	sea $z = a^{2^{i}+3}$. Entonces $xz = a^{2^{i+1}+4} \in L$ pues 
    	$2^{i+1}+4$ es claramente par, pero $yz = a^{2^{i}+2^{j}+5} \notin L$
    	ya que $2^{i}+2{j}+5$ es impar porque dos números pares más uno impar
    	da como resultado un número impar.\\
    	Por lo tanto, $L$ no es un lenguaje regular.\\
    \end{proof}
    
    % Lenguaje 3.
    \item \{$x$ $|$ $x$ tiene el mismo número de $a$ y $b$\}
    \begin{proof}[Demostración:] Primero tomamos dos cadenas cualesquiera, 
    sean $x = a^{i}$ y $y = a^{j}$, con $i \neq j$. Ahora, sea $z = b^{i}$.
    Entonces $xz = a^{i}b^{i} \in L$ pues $xz$ tiene el mismo número de
    $a$ y $b$ pero $yz = a^{j}b^{i} \notin L$ ya que $yz$ no tiene el mismo
    número de a y b.\\
    Por lo tanto, $L$ no es un lenguaje regular.\\ 
    \end{proof}
    \end{itemize}
    
  \end{enumerate}
\end{document}
