\documentclass[letterpaper,11pt]{article}

% Soporte para los acentos.
\usepackage[utf8]{inputenc}
\usepackage[T1]{fontenc}    
% Idioma español.
\usepackage[spanish,mexico, es-tabla]{babel}
% Soporte de símbolos adicionales (matemáticas)
\usepackage{multirow}
\usepackage{amsmath}		
\usepackage{amssymb}		
\usepackage{amsthm}
\usepackage{amsfonts}
\usepackage{latexsym}
% Modificamos los márgenes del documento.
\usepackage[lmargin=2cm,rmargin=2cm,top=2cm,bottom=2cm]{geometry}
% Soporte para dibujar con Tikz.
\usepackage{tikz}
\usetikzlibrary{automata, positioning, arrows}

% Información para el título
\title{Autómatas y Lenguajes Formales\\ Tarea 2}
\author{Rubí Rojas Tania Michelle}
\date{11 de octubre de 2018}

\begin{document}
    \maketitle
    \begin{enumerate}
        % Pregunta 1.
        \item Muestra que los lenguajes libres del contexto no son cerrados
        bajo intersección ni complemento.
        \begin{proof}
            Probaremos que los lenguajes libres del contexto no son cerrados 
            bajo la operación intersección usando un contraejemplo. Sean 
            $L_1 = \{a^{n}b^{n}c^{m} : m,n \geq 0\}$ y 
            $L_2 = \{a^{m}b^{n}c^{n} : m,n \geq 0\}$ lenguajes libres del 
            contexto. Entonces $L_1 \cap L_2 = \{a^{n}b^{n}c^{n} : n \geq 0\}$,
            el cual no es un lenguaje libre del contexto, por lo visto en
            clase. Por lo tanto, los lenguajes libres del contexto no son 
            cerrados bajo la intersección. \\
            Ahora, probaremos que los lenguajes libres del contexto no son 
            cerrados bajo la operación complemento. Si el complemento de un
            lenguaje libre del contexto fuera siempre libre del contexto
            entonces podríamos demostrar la cerradura de la intersección
            usando las leyes de De Morgan:
            \begin{center}
                $L_1 \cap L_2 = (L_1^{c} \cup L_2^{c})^{c}$
            \end{center}
            Pero como los lenguajes libres del contexto no son cerrados bajo
            la intersección, entonces tampoco pueden ser cerrados bajo 
            complemento.Por lo tanto, los lenguajes libres del contexto no son
            cerrados bajo el complemento.\\
        \end{proof}
        
        % Pregunta 2.
        \item Demuestra que si $M$ es un $AFD$, existe un $APN$ que simula 
        a $M$ aceptando por estado final y sin remover símbolos de la pila    
        durante su ejecución.
        \begin{proof}
            Supongamos que $M$ es un $AFD$. Lo que debemos hacer es que
            un $M_P$, que es un $APN$, imite el comportamiento de $M$. Si
            $M_P$ es un autómata de pila que acepta por estado final y no
            remueve ningún símbolo de la pila, entonces claramente $M_P$
            se comporta como un $AFD$. \\
            Formalmente, sea $M = (Q, \Sigma, \delta, q_0, F)$ un autómata finito
            determinista. Construimos \\
            $M_A = (Q, \Sigma, \varnothing,\delta^{'}, q_0, F)$ un autómata de pila,
            con $\delta^{'} = \{((p,u,\epsilon), (q, \epsilon)) :  (p, u, q) \in \delta\}$
            que acepta el mismo lenguaje. Es claro que con esta transición, el
            autómata de pila reconoce el mismo lenguaje que el AFD. Por lo tanto,
            $M_P$ simula a $M$ aceptando por estado final y sin remover símbolos.\\             
            
        \end{proof}
        
        \newpage
        % Pregunta 3.
        \item Da gramáticas libres del contexto que generen los siguientes
        lenguajes sobre el alfabeto $\sum = \{a,b\}$ :
        \begin{itemize}
            % Primer lenguaje.
            \item $\{w$ $|$ $w$ inicia y termina con el mismo símbolo\} \\
            \textit{Solución:} La gramática que genera al lenguaje es la
            siguiente:
            \begin{center}
                $S \rightarrow a$ $|$ $b$ $|$ $aTa$ $|$ $bTb$ \\
                $T \rightarrow \epsilon$ $|$ $aT$ $|$ $bT$
            \end{center}
            
            % Segundo lenguaje.
            \item $\{w$ $|$ $w$ es de longitud impar y el símbolo de enmedio 
            es $b$\} \\
            \textit{Solución:} La gramática que genera al lenguaje es la
            siguiente:
            \begin{center}
                $S \rightarrow aSa$ $|$ $bSb$ $|$ $aSb$ $|$ $bSa$ $|$ $b$
            \end{center}
           
            % Tercer lenguaje.
            \item $\{w$ $|$ $w$ tiene menos símbolos $a$ que símbolos $b$\} \\
            \textit{Solución:} La gramática que genera al lenguajes es la
            siguiente:
            \begin{center}
               $S \rightarrow bS$ $|$ $bT$ $|$ $TS$ \\
               $T \rightarrow \epsilon$ $|$ $aTb$ $|$ $bTa$ $|$ $TT$
            \end{center}
            
            % Cuarto lenguaje.
            \item $\{w$ $|$ $w$ tiene tantos símbolos $a$ como símbolos $b$\} \\
            \textit{Solución:} La gramática que genera al lenguajes es la
            siguiente:
            \begin{center}
               $S \rightarrow \epsilon$ $|$ $aSb$ $|$ $bSa$ $|$ $SS$
            \end{center}
            
            % Quinto lenguaje.
            \item $\{ww^{R}$ $|$ $w \in \sum^{*}\}$ \\
            \textit{Solución:} La gramática que genera al lenguajes es la
            siguiente:
            \begin{center}
               $S \rightarrow \epsilon$ $|$ $aSa$ $|$ $bSb$
            \end{center}
       
        \end{itemize}
        
        % Pregunta 4.
        \item Elige tres lenguajes del inciso anterior y construye autómatas de 
        pila que los reconozcan. 
        \begin{itemize}
            % Primer lenguaje
            \item  $\{w$ $|$ $w$ inicia y termina con el mismo símbolo\} \\
            \textit{Solución:}
            \begin{figure}[ht]
                \centering
                \begin{tikzpicture}[->,>=stealth',shorten >=1pt,auto,node 
                   distance=3cm, semithick]
                   % Asignamos los estados del autómata. 
                   \node[state, initial] (q0) {$q_0$};
                   \node[state, below of=q0] (q1) {$q_1$};
                   \node[state, accepting, right of=q1] (q2) {$q_2$};
                   % Dibujamos las transiciones del autómata.
                   \draw (q0) edge[above left] 
                   node[align=left]{$a, \epsilon \rightarrow a$ \\  
                   $b,\epsilon \rightarrow b$}(q1)
                   (q1) edge[loop below] 
                   node[align=left]{$a, \epsilon \rightarrow \epsilon$ \\
                   $b, \epsilon \rightarrow \epsilon$} (q1)
                   (q1) edge[above] node[align=left]{$a, a \rightarrow
                   \epsilon$ \\ $b, b \rightarrow \epsilon$} (q2)
                   (q0) edge[bend left] 
                   node[align=left]{$a, \epsilon \rightarrow \epsilon$ \\
                   $b, \epsilon \rightarrow \epsilon$} (q2);
               \end{tikzpicture}
           \end{figure}
           
           \newpage
           % Segundo lenguaje.
           \item $\{w$ $|$ $w$ es de longitud impar y el símbolo de enmedio 
            es $b$\} \\
           \textit{Solución:}
            \begin{figure}[ht]
                \centering
                \begin{tikzpicture}[->,>=stealth',shorten >=1pt,auto,node 
                   distance=2.8cm, semithick]
                   % Asignamos los estados del autómata. 
                   \node[state, initial] (q0) {$q_0$};
                   \node[state, below of=q0] (q1) {$q_1$};
                   \node[state, right of=q1] (q2) {$q_2$};
                   \node[state, accepting, above of=q2] (q3) {$q_3$};
                   % Dibujamos las transiciones del autómata.
                   \draw (q0) edge[above left] 
                   node{$\epsilon, \epsilon \rightarrow \$ $} (q1)
                   (q1) edge[loop below] 
                   node[align=left]{$a, \epsilon \rightarrow a$ \\ 
                   $b, \epsilon \rightarrow b$} (q1)
                   (q1) edge[above] 
                   node{$b, \epsilon \rightarrow \epsilon $} (q2)
                   (q2) edge[loop right] 
                   node[align=right]{$b, b \rightarrow \epsilon$ \\ 
                   $b, a \rightarrow \epsilon$ \\ 
                   $a, b \rightarrow \epsilon$ \\
                   $a, a \rightarrow \epsilon$} (q2)
                   (q2) edge[above left] 
                   node{$\epsilon, \$ \rightarrow \epsilon$ } (q3);
               \end{tikzpicture}
           \end{figure}
           
           \item $\{ww^{R}$ $|$ $w \in \sum^{*}\}$ \\
           \textit{Solución:}
           \begin{figure}[ht]
                \centering
                \begin{tikzpicture}[->,>=stealth',shorten >=1pt,auto,node 
                   distance=2.8cm, semithick]
                   % Asignamos los estados del autómata. 
                   \node[state,  accepting, initial] (q0) {$q_0$};
                   \node[state, right of=q0] (q1) {$q_1$};
                   \node[state, below of=q1] (q2) {$q_2$};
                   \node[state, accepting, left of =q2] (q3) {$q_3$};
                   % Dibujamos las transiciones del autómata.
                   \draw (q0) edge[above] 
                   node{$\epsilon, \epsilon \rightarrow \$ $} (q1)
                   (q1) edge[loop right] 
                   node[align=left]{$a, \epsilon \rightarrow a$ \\
                   $b, \epsilon \rightarrow b$} (q1)
                   (q1) edge[above right] node{$\epsilon, \epsilon \rightarrow
                   \epsilon$} (q2)
                   (q2) edge[loop right] 
                   node[align=left]{$a, a \rightarrow \epsilon$ \\
                   $b, b \rightarrow \epsilon$} (q2)
                   (q2) edge[above] node{$\epsilon, \$ \rightarrow
                   \epsilon$} (q3);
               \end{tikzpicture}
           \end{figure}
       \end{itemize}
       
       % Pregunta 5.
       \item El lenguaje \textit{Tak!} se define de la siguiente manera:
       \begin{center}
           T $\rightarrow$ Tak! | $\langle T \rangle$ | TT
       \end{center}
       Construye un autómata de pila sobre el alfabeto 
       $\sum = \{ \langle, \rangle, T, a, k, !\}$ que lo reconozca y describe 
       informalmente la ejecución con la cadena $\langle Tak!Tak! \rangle$.\\
       \textit{Solución:} Las transiciones del autómata de pila correspondiente
       están dadas en la tabla siguiente:
       \begin{center}
           \begin{tabular}{| c | c | c |}
               \hline
               1 & $(q_0, \epsilon, \epsilon)$ & $(q_1, T)$ \\
               2 & $(q_1, \epsilon, T)$ & $(q_1, Tak!)$ \\
               3 & $(q_1, \epsilon, T)$ & $(q_1, \langle T \rangle)$ \\
               4 & $(q_1, \epsilon, T)$ & $(q_1, TT)$ \\
               5 & $(q_1, Tak!, Tak!)$ & $(q_1, \epsilon)$ \\
               8 & $(q_1, \langle, \langle)$ & $(q_1, \epsilon)$ \\
               9 & $(q_1, \rangle, \rangle)$ & $(q_1, \epsilon)$ \\
               \hline
           \end{tabular}
       \end{center}
       
       \newpage
       El funcionamiento de este autómata de pila ante la cadena 
       $\langle Tak!Tak! \rangle$ aparece en la siguiente tabla:
       \begin{center}
           \begin{tabular}{ c | c | c }
               \hline
               Estado & Falta leer & Pila \\
               \hline
               $q_0$ & $\langle Tak!Tak! \rangle$ & $\epsilon$ \\
               $q_1$ & $\langle Tak!Tak! \rangle$ & $T$ \\
               $q_1$ & $\langle Tak!Tak! \rangle$ & $\langle TT \rangle$ \\
               $q_1$ & $ Tak!Tak! \rangle$ & $TT \rangle$ \\
               $q_1$ & $ Tak!Tak! \rangle$ & $Tak!T \rangle$ \\
               $q_1$ & $ Tak! \rangle$ & $ T \rangle$ \\
               $q_1$ & $ Tak! \rangle$ & $ Tak! \rangle$ \\
               $q_1$ & $ \rangle$ & $ \rangle$ \\
               $q_1$ & $\epsilon$ & $\epsilon$ \\
               \hline
           \end{tabular}
       \end{center}
      
       % Pregunta 6.
       \item Considera el lenguaje 
       $L = \{w \in \{0, 1\}^{*}$ $|$ $w^{R} = \overline{w}\}$ donde 
       $\overline{x}$ es la cadena $x$ con cada uno de sus bits invertidos, por 
       ejemplo $\overline{0101}= 1010$.
       \begin{itemize}
           % Pregunta 6.1
           \item Da un autómata de pila que reconozca $L$. \\
           \textit{Solución:}
           \begin{figure}[ht]
                \centering
                \begin{tikzpicture}[->,>=stealth',shorten >=1pt,auto,node 
                   distance=3cm, semithick]
                   % Asignamos los estados del autómata. 
                   \node[state, accepting, initial] (q0) {$q_0$};
                   \node[state, right of=q0] (q1) {$q_1$};
                   \node[state, right of=q1] (q2) {$q_2$};
                   \node[state, accepting, right of=q2] (q3) {$q_3$};
                   % Dibujamos las transiciones del autómata.
                   \draw (q0) edge[above] 
                   node{$\epsilon, \epsilon \rightarrow \$$}(q1)
                   (q1) edge[loop below] 
                   node[align=left]{$0, \epsilon \rightarrow 0$ \\
                   $1, \epsilon \rightarrow 1$} (q1)
                   (q1) edge[above] node[align=left]{$0, \epsilon \rightarrow
                   \epsilon$ \\ $1, \epsilon \rightarrow \epsilon$ \\
                   $\epsilon, \epsilon \rightarrow \epsilon$} (q2)
                   (q2) edge[loop below] node[align=left]{$0, 0 \rightarrow
                   \epsilon$ \\ $1, 1 \rightarrow \epsilon$} (q2)
                   (q2) edge[above] node{$\epsilon, \$ \rightarrow
                   \epsilon$}(q3);
               \end{tikzpicture}
           \end{figure}
           
           % Pregunta 6.2
           \item Da una gramática libre del contexto que genere $L$.\\
           \textit{Solución:} La gramática que genera al lenguaje es la
           siguiente:
           \begin{center}
            $S \rightarrow 0$ $|$ $1$ $|$ $0S0$ $|$ $1S1$ $|$ $\epsilon$
           \end{center}

       
       \end{itemize}
       
       % Pregunta 7.
       \item Elige un lenguaje del inciso 3 y da una gramática en forma normal
       de Chomsky que lo genere. Además, da una gramática en forma normal de 
       Griebach que lo genere. \\
       \textit{Solución:} Sea $L = \{w$ $|$ $w$ es de longitud impar y el
       símbolo de enmedio es $b$\} el lenguaje generado por la siguiente
       gramática:
       \begin{center}
            $S \rightarrow aSa$ $|$ $bSb$ $|$ $aSb$ $|$ $bSa$ $|$ $b$
       \end{center}
       La gramática en Forma Normal de Chomsky que lo genera es:
       \begin{center}
           $S \rightarrow TX$ $|$ $UY$ $|$ $TY$ $|$ $UX$ $|$ $U$ \\
           $T \rightarrow a$ \\
           $U \rightarrow b$ \\
           $X \rightarrow ST$ \\
           $Y \rightarrow SU$ \\
       \end{center}
       
       \newpage
       La gramática en Forma Normal de Griebach que lo genera es:
       \begin{center}
           $S \rightarrow aX$ $|$ $bY$ $|$ $aY$ $|$ $bX$ $|$ $b$ \\
           $T \rightarrow a$ \\
           $U \rightarrow b$ \\
           $X \rightarrow aXT$ $|$ $bYT$ $|$ $aYT$ $|$ $bXT$ $|$ $bT$ \\
           $Y \rightarrow aXU$ $|$ $bYU$ $|$ $aYU$ $|$ $bXU$ $|$ $bU$
       \end{center}

       % Pregunta 8.
       \item Muestra que los siguientes lenguajes no son libres del contexto:
       \begin{itemize}
           % Lenguaje 1.
           \item $\{a^{p}$ $|$ $p$ es un número primo\}
           \begin{proof} Veamos que $L$ no es un lenguaje libre del contexto.
               \begin{itemize}
                   \item[i)] Sea $k \in \mathbb{N}$.
                   \item[ii)] $z = a^{k}$, donde $k$ es un número primo que
                   es al menos $n + 2$, donde $n$ es la longitud del bombeo.
                   \item[iii)] $z = uvwxy$, $|vx| > 0$, $|vwx| \leq k$.
                   \item[iv)] Sea $i = n$. Si $|vx| = m$ y 
                   $z' = uv^{k-m}wx^{k-m}y$, entonces 
                   $|z'| = m(k-m) + m = (m + 1)(k - m)$. Tenemos que $m+1$
                   debe ser al menos $2$, por lo que $|vx| > 0$. Además,
                   $k - m$ debe ser al menos $2$ ya que $k$ es al menos
                   $n + 2$, y $|vx| \leq |vwx| \leq n$. Por lo anterior,
                   $|z'|$ no puede ser un número primo.


               \end{itemize}
               Por lo tanto, $L$ no es un lenguaje libre del contexto. \\
           \end{proof}
           
           % Lenguaje 2.
           \item $\{a^{2^{n}}$ $|$ $n \in \mathbb{N}\}$
           \begin{proof} Veamos que L no es un lenguaje libre del contexto.
               \begin{itemize}
                   \item[i)] Sea $k \in \mathbb{N}$.
                   \item[ii)] $z = a^{2^{k}}$.
                   \item[iii)] $z = uvwxy$, $|vx| > 0$, $|vwx| \leq k$.
                   \item[iv)] Sea $i = 2$. Si $v = a^{2^{n}}$ y
                   $x = a^{2^{m}}$ entonces sabemos que
                   $2^{n} + 2^{m} = |vx| \leq |vwx| \leq k$ y 
                   $2^{n} + 2^{m} = |vx| > 0$ por lo que 
                   $0 < 2^{n} + 2^{m} \leq k$. \\
                   Notemos que $|uv^{i}wx^{i}y| = 2^{k} + (2^{n} + 2^{m})
                   \leq 2^{k} + k < 2^{k} + 2^{k} = 2^{k+1}$ siempre y 
                   cuando $k > 0$, así al bombear la cadena $uv^{i}wx^{i}y$,
                   ésta tiene estrictamente una longitud entre $2^{k}$ y
                   $2^{k+1}$. Ninguna cadena de este tipo puede estar en 
                   el lenguaje, por lo que $uv^{i}wx^{i}y \notin L$.
                   
               \end{itemize}
               Por lo tanto, L no es un lenguaje libre del contexto. \\
           \end{proof}
           
           % Lenguaje 3.
           \item $\{w \in \{a,b,c\}^{*}$ $|$ $n_a(w) = n_b(w) = n_c(w)\}$
           \begin{proof} Veamos que $L$ no es un lenguaje libre del contexto. \\
               Si $L = \{w \in \{a,b,c\}^{*}$ $|$ $n_a(w) = n_b(w) = n_c(w)\}$
               entonces $L \cap \{a^{*}b^{*}c^{*}\} = \{a^{n}b^{n}c^{n}\}$. 
               Sabemos que la intersección entre un lenguaje libre del contexto
               ($L$) y un lenguaje regular ($a^{*}b^{*}c^{*}$) es un lenguaje
               libre del contexto. Pero sabemos que $\{a^{n}b^{n}c^{n}\}$ no es
               un lenguaje libre del contexto, contradicción. Por lo tanto,
               $L$ no es un lenguaje libre del contexto.\\
           \end{proof}
           
           \newpage
           % Lenguaje 4.
           \item $\{0^{n}1^{n}0^{n}1^{n} \in \{0, 1\}^{*}$ $|$
           $n \in \mathbb{N}\}$
           \begin{proof} Veamos que $L$ no es un lenguaje libre del contexto. 
               \begin{itemize}
                   \item[i)] Sea $k \in \mathbb{N}$.
                   \item[ii)] $z = 0^{k}1^{k}0^{k}1^{k}$.
                   \item[iii)] $z = uvwxy$, $|vx| > 0$, $|vwx| \leq k$.
                   \item[iv)] Entonces $vwx$ puede ser como los siguientes casos:
                   \begin{itemize}
                       % Primer caso.
                       \item $v \in 0^{k}$ y $x \in 0^{k}$. Esto no es posible
                       porque se infrige la regla de que $|vwx| \leq k$, ya que
                       $w$ consistiría de al menos $k$ números. Por lo tanto, $z \notin L$.
                       
                       \item $v \in 1^{k}$ y $x \in 1^{k}$. Esto no es posible
                       porque se infrige la regla de que $|vwx| \leq k$, ya que
                       $w$ consistiría de al menos $k$ números. Por lo tanto, $z \notin L$.
                       
                       % Segundo caso.
                       \item $v \in 0^{k}$ y $x \in 1^{k}$. Al hacer $i = 0$, 
                       dado que $|vx| >  0$, $a^{k}$ y/o en $1^{k}$ tendríamos menos
                       números que en $0^{k}$ y/o $1^{k}$, por lo que $z_0 = \notin L$.
                       
                       
                       % Tercer caso.
                       \item $v$ o $x$ ocupan posiciones sobre la frontera entre dos regiones.
                       Esto es, cualquiera de las dos (pero no ambas) consiste de $a...ab...b$;
                       al bombear la cadena para $i = 2$, se mezclarían las $a$es y $b$es y no
                       se respetarían las regiones, por lo que el resultado de bombear no estaría
                       en $L$.
                   \end{itemize}
                   
               \end{itemize}
               Por lo tanto, $L$ no es un lenguaje libre del contexto. \\
           \end{proof}
        \end{itemize}

        % Pregunta 9.
        \item Ejecuta detalladamente el algoritmo $CYK$ con la gramática:
        \begin{center}
            $S \rightarrow AB$ \\
            $A \rightarrow a$ \\ 
            $B \rightarrow AB$ $|$ $b$
        \end{center}
        y con la cadena $w = abbb$.\\ \\
        \textit{Solución:} Dada la cadena $w = abbb$ y la gramática $G$, veamos
        si $w \in L(G)$. \\
        Sea $|_{0}$ $a$ $|_1$ $b$ $|_2$ $b$ $|_3$ $b$ $|_4$ la cadena $w$
        dividida en subíndices, esto para tener una ayuda visual para ejecutar 
        el algoritmo.
        \begin{itemize}
            % Primer paso.
            \item[i)] Inicializamos la tabla con la longitud de la cadena, 
            $n = |w| = 4$. Comenzamos con las subcadenas de longitud $1$. Sea
            $C = X_{i, i+1}$. Si existe una producción $A \rightarrow C$, 
            entonces escribimos $A$ en la entrada $i, i+1$ de la tabla.
            \begin{center}
                \begin{tabular}{c | c | c| c | c}
                     0   \\
                     A & 1  \\
                     & B & 2 \\
                     & & B & 3 \\
                     & & & B & 4\\
                     \hline
                \end{tabular}
            \end{center}
            
            % Segundo paso.
            \item[ii)] Seguimos con las producciones para subcadenas de 
            longitud dos. La subcadena $X_{i, i+2}$ se divide en dos cadenas:
            $X_{i, i+i}$ y $X_{i+1, i+2}$. Si hay una producción 
            $Z \rightarrow XY$ tal que $X = (i, i+1)$ y $Y =(i+1, i+2)$
            entonces colocamos $Z$ en la entrada (i, i+2).
            \begin{center}
                \begin{tabular}{c | c | c| c | c}
                     0   \\
                     A & 1  \\
                     S, B & B & 2 \\
                     & $\varnothing$ & B & 3 \\
                     & & $\varnothing$ & B & 4\\
                     \hline
                \end{tabular}
            \end{center}
            
            \newpage
         \item[iii)] El siguiente paso son las producciones para las subcadenas
         de longitud tres. La subcadena $X_{i, i+3}$ sigue instrucciones 
         análogas al paso dos. Tenemos sólo dos casos:
         $X_{0,3} = X_{0,1}X_{1,3} = X_{0,2}X_{2,3}$ y
         $X_{1,4} = X_{1,2}X_{2,4} = X_{1,3}X_{3,4}$.
         \begin{center}
                \begin{tabular}{c | c | c| c | c}
                     0   \\
                     A & 1  \\
                     S, B & B & 2 \\
                     $\varnothing$ & $\varnothing$ & B & 3 \\
                     & $\varnothing$ & $\varnothing$ & B & 4\\
                     \hline
                \end{tabular}
            \end{center}
         \item[iv)] Concluimos con las producciones para las sucadenas de 
         longitud cuatro. La subcadena $X_{i, i+4}$ sigue las instrucciones
         análogas al paso dos. Tenemos únicamente un caso:\\
         $X_{0,4} = X_{0,3}X_{3,4} = X_{0,1}X_{1,4}$
         \begin{center}
                \begin{tabular}{c | c | c| c | c}
                     0   \\
                     A & 1  \\
                     S, B & B & 2 \\
                     $\varnothing$ & $\varnothing$ & B & 3 \\
                     $\varnothing$ & $\varnothing$ & $\varnothing$ & B & 4\\
                     \hline
                \end{tabular}
            \end{center}
        \end{itemize}
        Como el símbolo inicial no aparece en la entrada (a,n) entonces podemos
        concluir que $w \notin L(G)$. 
    \end{enumerate}
\end{document}
